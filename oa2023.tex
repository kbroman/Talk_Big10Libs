\documentclass[aspectratio=169,12pt,t]{beamer}
\usepackage{graphicx}
\setbeameroption{hide notes}
\setbeamertemplate{note page}[plain]
\usepackage{listings}

\input{header.tex}

%%%%%%%%%%%%%%%%%%%%%%%%%%%%%%%%%%%%%%%%%%%%%%%%%%%%%%%%%%%%%%%%%%%%%%
% end of header
%%%%%%%%%%%%%%%%%%%%%%%%%%%%%%%%%%%%%%%%%%%%%%%%%%%%%%%%%%%%%%%%%%%%%%

% title info
\title{The culture of open scholarship}
\author{\href{https://kbroman.org}{Karl Broman}}
\institute{Biostatistics \& Medical Informatics \\ Univ.\ Wisconsin{\textendash}Madison}
\date{\href{https://kbroman.org}{\tt \scriptsize \color{foreground} kbroman.org}
\\[-5pt]
\href{https://github.com/kbroman}{\tt \scriptsize \color{foreground} github.com/kbroman}
\\[-5pt]
\href{https://fosstodon.org/@kbroman}{\tt \scriptsize \color{foreground} @kbroman@fosstodon.org}
\\[2pt]
\scriptsize {\lolit Slides:} \href{https://bit.ly/broman2023}{\tt \scriptsize
  \color{foreground} bit.ly/broman2023}
}


\begin{document}

% title slide
{
\setbeamertemplate{footline}{} % no page number here
\frame{
  \titlepage

  \vfill \hfill \includegraphics[height=6mm]{Figs/cc-zero.png} \vspace*{-1cm}


  \note{
    These are slides for a 10-15 min talk on open scholarship for the Big 10
    Academic Alliance libraries, in Jan 2023.

    Slides (pdf): https://kbroman.org/Talk\_Big10Libs/oa2023.pdf

    Slides with notes (pdf): https://kbroman.org/Talk\_Big10Libs/oa2023\_notes.pdf

    Source: https://github.com/kbroman/Talk\_Big10Libs
  }
} }



\begin{frame}[c]{}

\hfill \begin{minipage}{108mm}

{\Large \color{title}

  open access

\bigskip

  open educational resources

\bigskip

  open source

\bigskip

  open science

}
\end{minipage}

  \note{
      In thinking about Open Scholarship, I'm generally thinking of four
      things: open access publications, open educational resources,
      open source software, and open science (by which I mean open
      data, methods, and materials).

      I'm going to focus mostly on Open Access publications, and on
      academic scientists.

      Concerns in the humanities can be quite different, and I'm going
      to focus on what I know best, which is the situation in science.
  }
\end{frame}


\begin{frame}{About me}

  \vspace{-12pt}

  \bbi
\item Applied statistician working in genetics
\item Write \& support many open-source software packages
\item Co-author on 170 papers and 1 book
\item Reviewer for 90 different journals
\item Formerly
  \bi
\item Associate Editor and Senior Editor at \emph{Genetics}
\item Associate Editor at \emph{Biostatistics}
\item Associate Editor at \emph{Journal of the Americal Statistical
Association}
\item Academic Editor at \emph{PeerJ}
\item Editorial Board member of \emph{BMC Biology}
  \ei
  \ei

  \vspace{30pt}
  \hfill
  \href{https://kbroman.org/broman_cv.pdf}{\tt \footnotesize \lolit kbroman.org/broman{\_}cv.pdf}

  \note{
     I thought I should say a bit about my experience with the
     publication process.

     My research spans genetics and statistics, but the majority of my
     publications are in the genetics or biology literature; I only
     have a couple of papers in statistics journals.

     I've reviewed for a lot of different journals, and I spent a
     decade as an editor for the society journal, Genetics.
  }
\end{frame}



\begin{frame}[c]{}

  \vspace{24pt}

  \centerline{\fbox{\parbox[c]{\textwidth}{
  \figw{Figs/dataorg_paper.png}{1.0}
  }}}

  \vfill
  \hfill
  \href{https://doi.org/gdz6cm}{\tt \footnotesize \lolit doi.org/gdz6cm}

  \note{
    I thought I'd start with a story about the paper of mine.

    This isn't a typical research paper, but really more of a
    tutorial, or really a screed.
  }
\end{frame}


\begin{frame}[c]{}
  \figh{Figs/dataorg_fig5c.png}{0.9}

  \note{
    The paper concerns how to organize data in spreadsheets. And
    really that spreadsheets shouldn't be organized like this example,
    but rather as a rectangle with the columns being the measured
    variables and one row per subject, and a single header row.
  }
\end{frame}


\begin{frame}[c]{}
  \figh{Figs/amstat_mostread.png}{0.9}

  \note{
    It was published in the American Statistician, a society journal,
    and is third-most downloaded paper in that journal, after two
    papers about P-values.
  }
\end{frame}



\begin{frame}[c]{}
  \vspace{8pt}

  \centerline{\fbox{\parbox[c]{0.72\textwidth}{
  \figh{Figs/dataorg_website.png}{0.9}
  }}}

  \vfill
  \hfill
  \href{https://kbroman.org/dataorg}{\tt \footnotesize \lolit kbroman.org/dataorg}

  \note{
    Before it was a paper, it was just a website. It's one of several
    short tutorials that I've written, about things related to
    reproducible research and open science.
  }
\end{frame}


\begin{frame}[c]{}
  \vspace{8pt}

  \centerline{\fbox{\parbox[c]{0.8\textwidth}{
  \figh{Figs/dataorg_preprint.png}{0.9}
  }}}

  \vfill
  \hfill
  \href{https://doi.org/10.7287/peerj.preprints.3183v2}{\tt \footnotesize \lolit doi.org/10.7287/peerj.preprints.3183v2}

  \note{
    In addition, the submitted manuscript is openly available at PeerJ
    Preprints.
  }
\end{frame}


\begin{frame}[c]{}
  \figh{Figs/dataorg_invoice.png}{0.9}

  \note{
    Nevertheless, I paid nearly \$3000 to have the published paper
    available open access.

    I struggled with the decision of whether to pay this fee. But I'm
    glad that I did.

    I think audience for this paper was made much more widespread by
    being a formal paper (rather than a preprint, and before that
    basically a blog post).
  }
\end{frame}


\begin{frame}[c]{}
  \vspace{8pt}

  \centerline{\fbox{\parbox[c]{0.5\textwidth}{
  \figh{Figs/dataorg_funfacts.png}{0.9}
  }}}

  \vfill
  \hfill
  \href{https://bit.ly/3sIRtVY}{\tt \footnotesize \lolit bit.ly/3sIRtVY}

  \note{
    If you want to more about this paper, see my twitter thread on 10
    fun facts about the paper.
  }
\end{frame}


\begin{frame}[c]{Personal timeline}

\figw{Figs/timeline.pdf}{1.0}

  \note{
  }
\end{frame}


\begin{frame}{What has changed?}

  \bbi
\item Rise of preprints in biology and medicine
\item Rise of \emph{Nature Communications}
\item PubMed Central: expansion, with no embargo
\item No longer stigma on OA
\item Emphasis on computational reproducibility
  \ei

  \note{
  }
\end{frame}


\begin{frame}{What hasn't changed?}

  \bbi
\item Attachment to Impact Factor
\item Attachment to Glam Journals
\item Journal and conference spam
\item The 20 open access enthusiasts on campus
\item Researchers don't read much
  \ei

  \note{
    The journal Genetics recently sent out a newsletter that
    included an announcement of new associate editors. The bio for one
    of them had ``...including more than 30 articles in Nature, Science,
    Cell, and Nature Genetics.''
  }
\end{frame}


\begin{frame}{Culture of open scholarship}

  \bbi
\item Community before individual
\item Sharing makes better science
  \bi
\item Data, methods, software, materials, manuscripts
  \ei
  \ei

  \note{
  }
\end{frame}





\begin{frame}[c]{Traditional scholarship}

  \Large
  \centering

  What's in it for me?


  \note{
  }
\end{frame}





\begin{frame}{Barriers to open scholarship}

  \bbi
\item Focus on glamour/prestige
\item Apathy
\item Ignorance
\item Concern about being scooped
\item Cost
\item Funding of scientific societies
  \ei

  \note{
  }
\end{frame}



\begin{frame}{How to persuade?}

  \bbi
\item Moral arguments
\item Advantages for the author
\item Institution policies
\item Government policies
  \ei

  \note{
     How to turn a successful capitalist into a socialist?
  }
\end{frame}






\begin{frame}[c]{Privilege}

  \centering \Large
  white, male, US-born full professor

  in cargo shorts and a hoodie

  whose father was a university professor

  \onslide<2>{

    \bigskip

    \vhilit
    credentials seldom questioned
  }



  \note{
    It bugs me to no end when senior faculty suggest solutions like
    "just remove journal titles from our CVs" and "stop sending papers
    to the big three journals" as these aren't real solutions for most
    people. Journal titles are like the schools people attended; we
    can wish they don't matter, but they do.
  }
\end{frame}


\begin{frame}{Questions}

  \bbi
\item How to relax reliance on journal prestige?
\item How to support junior faculty to be open scholars?
\item How to reorganize the way publishing is funded?
\item How to persuade researchers to care?
  \ei

  \note{
  }
\end{frame}





\end{document}
