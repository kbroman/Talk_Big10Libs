\documentclass[aspectratio=169,12pt,t]{beamer}
\usepackage{graphicx}
\setbeameroption{hide notes}
\setbeamertemplate{note page}[plain]
\usepackage{listings}

\input{header.tex}

%%%%%%%%%%%%%%%%%%%%%%%%%%%%%%%%%%%%%%%%%%%%%%%%%%%%%%%%%%%%%%%%%%%%%%
% end of header
%%%%%%%%%%%%%%%%%%%%%%%%%%%%%%%%%%%%%%%%%%%%%%%%%%%%%%%%%%%%%%%%%%%%%%

% title info
\title{The culture of open scholarship}
\author{\href{https://kbroman.org}{Karl Broman}}
\institute{Biostatistics \& Medical Informatics \\ Univ.\ Wisconsin{\textendash}Madison}
\date{\href{https://kbroman.org}{\tt \scriptsize \color{foreground} kbroman.org}
\\[-5pt]
\href{https://github.com/kbroman}{\tt \scriptsize \color{foreground} github.com/kbroman}
\\[-5pt]
\href{https://fosstodon.org/@kbroman}{\tt \scriptsize \color{foreground} @kbroman@fosstodon.org}
\\[2pt]
\scriptsize {\lolit Slides:} \href{https://bit.ly/broman2023}{\tt \scriptsize
  \color{foreground} bit.ly/broman2023}
}


\begin{document}

% title slide
{
\setbeamertemplate{footline}{} % no page number here
\frame{
  \titlepage

  \vfill \hfill \includegraphics[height=6mm]{Figs/cc-zero.png} \vspace*{-1cm}


  \note{
    These are slides for a 10-15 min talk on open scholarship for the Big 10
    Academic Alliance libraries, in Jan 2023.

    Slides (pdf): https://kbroman.org/Talk\_Big10Libs/oa2023.pdf

    Slides with notes (pdf): https://kbroman.org/Talk\_Big10Libs/oa2023\_notes.pdf

    Source: https://github.com/kbroman/Talk\_Big10Libs
  }
} }



\begin{frame}[c]{}

\hfill \begin{minipage}{108mm}

{\Large \color{title}

  open access

\bigskip

  open educational resources

\bigskip

  open source

\bigskip

  open science

}
\end{minipage}

  \note{
      In thinking about Open Scholarship, I'm generally thinking of four
      things: open access publications, open educational resources,
      open source software, and open science (by which I mean open
      data, methods, and materials).

      I'm going to focus mostly on Open Access publications, and on
      academic scientists.

      Concerns in the humanities can be quite different, and I'm going
      to focus on what I know best, which is the situation in science.
  }
\end{frame}


\begin{frame}{About me}

  \vspace{-12pt}

  \bbi
\item Applied statistician working in genetics
\item Write \& support many open-source software packages
\item Co-author on 170 papers and 1 book
\item Reviewer for 90 different journals
\item Formerly
  \bi
\item Associate Editor and Senior Editor at \emph{Genetics}
\item Associate Editor at \emph{Biostatistics}
\item Associate Editor at \emph{Journal of the Americal Statistical
Association}
\item Academic Editor at \emph{PeerJ}
\item Editorial Board member of \emph{BMC Biology}
  \ei
  \ei

  \vspace{30pt}
  \hfill
  \href{https://kbroman.org/broman_cv.pdf}{\tt \footnotesize \lolit kbroman.org/broman{\_}cv.pdf}

  \note{
     I thought I should say a bit about my experience with the
     publication process.

     My research spans genetics and statistics, but the majority of my
     publications are in the genetics or biology literature; I only
     have a couple of papers in statistics journals.

     I've reviewed for a lot of different journals, and I spent a
     decade as an editor for the society journal, Genetics.
  }
\end{frame}



\begin{frame}[c]{}

  \vspace{24pt}

  \centerline{\fbox{\parbox[c]{\textwidth}{
  \figw{Figs/dataorg_paper.png}{1.0}
  }}}

  \vfill
  \hfill
  \href{https://doi.org/gdz6cm}{\tt \footnotesize \lolit doi.org/gdz6cm}

  \note{
    I thought I'd start with a story about the paper of mine.

    This isn't a typical research paper, but really more of a
    tutorial, or really a screed.
  }
\end{frame}


\begin{frame}[c]{}
  \figh{Figs/dataorg_fig5c.png}{0.9}

  \note{
    The paper concerns how to organize data in spreadsheets. And
    really that spreadsheets shouldn't be organized like this example,
    but rather as a rectangle with the columns being the measured
    variables and one row per subject, and a single header row.
  }
\end{frame}


\begin{frame}[c]{}
  \figh{Figs/amstat_mostread.png}{0.9}

  \note{
    It was published in the American Statistician, a society journal,
    and is third-most downloaded paper in that journal, after two
    papers about P-values.
  }
\end{frame}



\begin{frame}[c]{}
  \vspace{8pt}

  \centerline{\fbox{\parbox[c]{0.72\textwidth}{
  \figh{Figs/dataorg_website.png}{0.9}
  }}}

  \vfill
  \hfill
  \href{https://kbroman.org/dataorg}{\tt \footnotesize \lolit kbroman.org/dataorg}

  \note{
    Before it was a paper, it was just a website. It's one of several
    short tutorials that I've written, about things related to
    reproducible research and open science.
  }
\end{frame}


\begin{frame}[c]{}
  \vspace{8pt}

  \centerline{\fbox{\parbox[c]{0.8\textwidth}{
  \figh{Figs/dataorg_preprint.png}{0.9}
  }}}

  \vfill
  \hfill
  \href{https://doi.org/10.7287/peerj.preprints.3183v2}{\tt \footnotesize \lolit doi.org/10.7287/peerj.preprints.3183v2}

  \note{
    In addition, the submitted manuscript is openly available at PeerJ
    Preprints.
  }
\end{frame}


\begin{frame}[c]{}
  \figh{Figs/dataorg_invoice.png}{0.9}

  \note{
    Nevertheless, I paid nearly \$3000 to have the published paper
    available open access.

    I struggled with the decision of whether to pay this fee. But I'm
    glad that I did.

    I think audience for this paper was made much more widespread by
    being a formal paper (rather than a preprint, and before that
    basically a blog post).
  }
\end{frame}


\begin{frame}[c]{}
  \vspace{8pt}

  \centerline{\fbox{\parbox[c]{0.5\textwidth}{
  \figh{Figs/dataorg_funfacts.png}{0.9}
  }}}

  \vfill
  \hfill
  \href{https://bit.ly/3sIRtVY}{\tt \footnotesize \lolit bit.ly/3sIRtVY}

  \note{
    If you want to more about this paper, see my twitter thread on 10
    fun facts about the paper.
  }
\end{frame}


\begin{frame}[c]{Personal timeline}

\figw{Figs/timeline.pdf}{1.0}

  \note{
    My academic career is basically coincident with the history of
    open access. This is my personal timeline, along with key events in
    the history of open access. I've worked on the editorial boards of
    a half dozen journals, but I include here just my work for the journal
    Genetics, as that work was most substantial and formative, for me.

    The beginnings of OA are basically the beginnings of the internet.
    The Journal of Statistical Software began in 1996 and has been
    online-only, open access, and free, with no APCs.

    The start of Creative Commons and PLOS are, to me, the start of
    the broader OA movement. The 2007 NIH policy requiring that funded
    manuscripts be deposited in PubMed Central was both exciting and
    disappointing (disappointing for the one-year embargo).

    The connection between OA and predatory publishers, and the
    initiation of Beall's infamous list, is my recent than I
    remembered.

    The start of PeerJ, eLife, and bioRxiv marked a second period of hope
    and disappointment. medRxiv and the COVID-19 pandemic brought a
    third wave of change.
  }
\end{frame}


\begin{frame}{What is new?}

  \bbi
\item Rise of preprints in biology and medicine
\item Rise of \emph{Nature Communications}
\item PubMed Central: expansion, with no embargo
\item No longer stigma on OA
\item Emphasis on computational reproducibility
  \ei

  \note{
    There have been a number of new developments in the last 5 or so
    years. The use of preprints has really taken off, particularly
    with the COVID-19 pandemic. The bioRxiv preprint repository had
    become quite popular in computational biology, but not its use
    seems much more broad, and biomedical research generally has
    finally begun to embrace preprints.

    At the same time, Nature Communications and related publications
    like Scientific Reports seem to be siphoning away papers from the
    PLOS and society journals in the biological sciences. I think this
    is due to the sparkle of Nature plus the ease of transfer after
    rejection by one of the glam Nature journals. The APCs are
    jaw-dropping, but researchers don't seem bothered.

    It looks like the PubMed Central idea will be expanded to all
    government-funded work, and with no embargo. That could really
    shake up journals' approaches to funding.

    The stigma on open access, in which OA was equated with low
    quality predatory publishers, has largely disappeared.

    A related development has been an increased emphasis on
    computational reproducibility in the sciences, which has led to
    broader adoption of openness in science.
  }
\end{frame}


\begin{frame}{What isn't new?}

  \bbi
\item Attachment to Journal Impact Factor
\item Attachment to Glam Journals
\item Journal and conference spam
\item The 20 open access enthusiasts on campus
\item Researchers don't read much
  \ei

  \note{
    Still, people focus on journal impact factors and glam journals.
    People complain about the focus on impact factors and then turn
    around and say things like ``They have 3 Nature papers'' when
    evaluating job candidates.

    The journal Genetics recently sent out a newsletter that
    included an announcement of new associate editors. The bio for one
    of them had ``...including more than 30 articles in Nature, Science,
    Cell, and Nature Genetics.''

    OA no longer has the stigma of predatory publishing, but it seems
    like 95\% of the email I receive is journal or conference spam.

    And while OA publishing has broadened, it still seems like most
    academic researchers don't much care. Hold a forum on open access
    publishing, and you'll likely be talking to the same 20 people as
    the last time.

    And researchers don't read much, which is the main reason to focus
    on Impact Factors. It's hard to evaluate the work itself; easier
    to just evaluate the reputation of the journal in which it
    appeared. As researchers have become increasingly specialized,
    it's become ever harder to evaluate our colleagues' work.
  }
\end{frame}


\begin{frame}{Culture of open scholarship}

  \bbi
\item Community before individual
\item Sharing makes better science
  \bi
\item Data, methods, software, materials, manuscripts
  \ei
  \ei

  \note{
     The culture of open scholarship generally places the needs of the
     community before the needs of an individual: be willing to make
     some short-term personal sacrifices in order to achieve larger,
     long-term benefits for the community.

     A central idea is that early and broad sharing of data, methods,
     and results will make for better science. We can't anticipate all
     possible uses of the data we generate. By making our readily
     available to others, in a form that is inter-operable with
     others' data, science as a whole will advance more rapidly.

     We should focus on solving problems and gaining knowledge, above
     getting credit.
  }
\end{frame}





\begin{frame}[c]{Traditional scholarship}

  \Large
  \centering

  What's in it for me?


  \note{
     But that view, of community before individual, appears rare.

     The modern approach to science is centered on advancement of an
     individual's research group. Collaboration is useful if it
     advances the individual's career, and not otherwise.

     While this is a gross simplification, it is also a good first
     approximation, and is useful for thinking about strategies to get
     university faculty to change their behavior.

     I can't stand the word ``incentivize,'' but that seems to be the
     way universities operate.
  }
\end{frame}





\begin{frame}{Barriers to open scholarship}

  \bbi
\item Focus on glamour/prestige
\item Apathy
\item Ignorance
\item Concern about being scooped
\item Cost
\item Funding of scientific societies
  \ei

  \note{


  }
\end{frame}



\begin{frame}{How to persuade?}

  \bbi
\item Moral arguments
\item Advantages for the researcher
\item Institution policies
\item Government policies
  \ei

  \note{
    How to turn a successful capitalist into a socialist?

    While we might think that we can point to the university's mission
    and ideals, talking about the Wisconsin Idea and Sifting and
    Winnowing and such, in practice it appears ineffective.

    More successful is to persuade by appealing to personal benefits
    that accompany open scholarship, for example that OA publications
    are more widely read and cited.

    I'm inclined to think that real change will come from top-down
    policies that require openness. Funding agencies recognize that
    they will get more from their investments if they require sharing
    of data and research products. I wish this weren't necessary, but
    it seems to be the case.
  }
\end{frame}






\begin{frame}[c]{Privilege}

  \centering \Large
  white, male, US-born full professor

  in cargo shorts and a hoodie

  whose father was a university professor

  \onslide<2>{

    \bigskip

    \vhilit
    credentials seldom questioned
  }



  \note{
    I should point out the considerable advantages that I've had, and
    that my situation gives me considerable flexibility in adopting
    a completely open approach in my scholarship. When proposing
    solutions, we need to recognize the very different situations
    experienced by junior and senior faculty, and between black women
    and white men.

    It bugs me to no end when senior faculty suggest solutions like
    "just remove journal titles from our CVs" and "stop sending papers
    to the big three journals" as these aren't real solutions for most
    people. Journal titles are like the schools people attended; we
    can wish they don't matter, but they do.
  }
\end{frame}


\begin{frame}{Questions}

  \bbi
\item How to relax reliance on journal prestige?
\item How to support junior faculty to be open scholars?
\item How to reorganize the way publishing is funded?
\item How to persuade researchers to care?
  \ei

  \note{
    I'll end with some of my own questions.

    I'm sure you can sense my frustration with the state of academic
    science. Looking back, we have seen some important progress
    towards openness of data and publications, achieved through both
    bottom-up innovation and top-down regulation.

    But the continued focus on glam journals and journal impact
    factors seems hard to break. The ever-increasing specialization of
    our research makes it ever more difficult to evaluate others'
    work, and so we continue to focus on short-cuts like journal
    impact factors.
  }
\end{frame}





\end{document}
